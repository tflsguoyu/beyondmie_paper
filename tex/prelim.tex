\section{Preliminaries}
\label{sec:prelim}
%
We now briefly revisit the basics on first principles of (classical) light transport theory based on Maxwell electromagnetism. Table~\ref{tb:symbols} summarize the symbols using along the paper.
\begin{table}[t]
    \caption{Symbols used along the paper.}
    \label{tb:symbols}
    \footnotesize
    %
    \addtolength{\tabcolsep}{-1pt}
    \renewcommand{\arraystretch}{1.1}
    \begin{tabular}{cl}
        \textbf{Symbol}   & \textbf{Definition} \\ 
        \toprule
        $\px\in\Real^3$ & Position \\
        $\dpx\in\Sphere$ & Direction to $\px$ \\
        $\tpx\in\Real$ & Distance \\
        \hline
        $\sPermittivity(\px)$ & Permittivity \\
        $\sPermeability(\px)$ & Permeability \\
        $\sFreq$ & Wave angular frequency [s$^{-1}$] \\
        $\lambda=2\pi\sFreq^{-1}$ & Wavelength [m] \\
        $k(\px)=\sFreq\sqrt{\sPermittivity(\px)\sPermeability(\px)}$ & Wavenumber at $\px$\\
        $\sIOR(\px)=k_2(\px)/k_1$ & Relative refractive index at $\px$ \\
        \hline
        $\MField(\px)$ & Magnetic field at $\px$ \\
        $\EField(\px)$   & Electric field at $\px$~\eqref{eq:vri}  \\
        $\IncEField(\px)$ & Incident electric field $\px$\\
        $\ScaEField(\px)$ & Scattered electric field at $\px$~\eqref{eq:vri}\\ 
        $\EField_0$ & Amplitude of a planar electric field \\
        $\ScaEField_1(\dpx)$ & Far-field angular distribution of the scattered radiation  \\
        \hline
        $\sGreen$ & Free-space dyadic Green's function~\eqref{eq:greenfunc} \\
        $\dyad{T}$ & Dyad transition operator~\eqref{eq:dyadtransition}\\
        $\sGreenProp(\dw,\px)$ & Planar field scalar propagator \\
        \hline
        \hline
        $V_i$ & Volume suspended by particle/cluster $i$ \\
        $\Px_i\in\Real^3$ & Representative position of particle/cluster $i$ \\
        $\dPx_{ij}\in\Sphere$ & Direction from $\Px_j$ to $\Px_i$\\
        $\tPx_{ij}\in\Real$ & Distance from $\Px_j$ to $\Px_i$\\
        $\Ncls$ & Number of particles in a cluster \\
        \hline
        $\ScaEField_i(\px)$ & Scattered field of $\px\in V_i$~\eqref{eq:foldylax}\\
        $\EField_i(\px)$ & Exciting field in $\px\in V_i$ \\
        $\ExcEField_{ij}(\px)$ & Partial exciting field in $\px\in V_i$ from particle $j$~\eqref{eq:excfield} \\
        $\ScaDyad_i^\text{near}(\dwi,\px)$ & Near-field scattering dyad of particle/cluster $i$~\eqref{eq:scatdyad_near} \\
        $\ScaDyad_i(\dwi,\dws)$ & Far-field scattering dyad of particle/cluster $i$~\eqref{eq:farscatdyad} \\
        \hline
        \hline
        $\cT(\dwi),\cS(\dwi)$ & Extinction~\eqref{eq:crosstcluster} and scattering~\eqref{eq:crossscluster} cross-sections [m$^{2}$]\\
        $\phase(\dwi,\dws)$ & Phase function~\eqref{eq:phasecluster} [sr$^{-1}$]\\
        $\rho$ & Particles density [m$^{-3}$] \\
        $\sigT(\dwi),\sigS(\dwi)$ & Extinction~\eqref{eq:sigmatcluster} and scattering~\eqref{eq:sigmascluster} coefficients [m$^{-1}$]\\
        \bottomrule
    \end{tabular}
\end{table}

\subsection{Electromagnetic Scattering}
\label{ssec:prelim_maxwells}
%
The propagation of a time-harmonic monochromatic electromagnetic field with frequency $\sFreq$ is defined by the Maxwell curl equations as
\begin{equation}
    \begin{aligned}
        \Curl\EField(\px) &= \img\,\sFreq\,\sPermeability(\px)\,\MField(\px),\\
        \Curl\MField(\px) &= \img\,\sFreq\,\sPermittivity(\px)\,\EField(\px),
    \end{aligned}
    \label{eq:maxwell}
\end{equation}
%
where $\Curl\cdot$ is the curl operator; $\EField(\px)$ and $\MField(\px)$ indicate, respectively, the (vector-valued) electric and magnetic fields at $\px$; $\sPermeability(\px)$ and $\sPermittivity(\px)$ denote the (scalar-valued) magnetic permeability and electric permittivity at $\px$, respectively; and $\img := \sqrt{-1}$ is the imaginary unit.

Assuming a non-magnetic medium satisfying $\sPermeability(\px) = \sPermeability_0$ with $\sPermeability_0$ being the magnetic permeability of a vacuum, \Eq{eq:maxwell} reduces to the electric field wave equation
%
\begin{equation}
    \nabla^2\times\EField(\px) - k^2\,\EField(\px) = 0,
    \label{eq:efieldwave}
\end{equation}
%
where $k(\px) = \sFreq\sqrt{\sPermittivity(\px)\sPermeability_0}$ is the medium's wave number at $\px$. 

We now assume an infinite homogeneous isotropic medium with permittivity $\sPermittivity_1$, filled with scatterers bounded by a finite disjoint region $V$, with potentially inhomogeneous permittivity $\sPermittivity_2(\px)$. Under this assumption, we can solve \Eq{eq:efieldwave} by expressing it as the \emph{volume integral equation} (see \S 3.1 of Mishchenko's work~\shortcite{mishchenko2006multiple} for a step-by-step derivation) as the sum of the incident field $\IncEField(\px)$ and the scattered field $\ScaEField(\px)$ due to inhomogeneities in the medium in the form of scatterers:
%
\begin{align}
    \EField(\px) & = \IncEField(\px) + \ScaEField(\px) \\
    & =\IncEField(\px) + k_1^2\,\int_V [\sIOR^2(\px')-1] \,\sGreen(\px,\px') \,\EField(\px') \diff{\px'},
    \label{eq:vri}
\end{align}
%
with $k_1$ the wave number at the hosting medium, $\sIOR(\px) = \nicefrac{k_2(\px)}{k_1}$ the index of refraction of the interior regions $V$ with respect to the hosting medium, and $\sGreen(\px,\px')$ the free-space dyadic Green's function defined as:
%
\begin{equation}
    \sGreen(\px,\px') = \left(\sIdDyad + k_1^{-2}\,\nabla\otimes\nabla\right) \frac{\exp(\img\,k_1 \,|\px-\px'|)}{4\pi\,|\px-\px'|},
    \label{eq:greenfunc}
\end{equation}
%
where $\sIdDyad$ is the identity dyad, and $. \otimes .$ denotes the dyadic product of two vectors. Intuitively, \Eq{eq:vri} models the scattering field as the superposition of the spherical wavelets resulting from a change of permitivitty (i.e. with $\sIOR(\px')\neq1$). Note also the recursive nature of \Eq{eq:vri}; we will deal with this recursivity in the following section, computing $\ScaEField(\px)$ as a function of the incident field $\IncEField(\px)$. 

\begin{figure*}
\centering
  \def\svgwidth{.5\columnwidth}
  \input{img/scheme/fig1.pdf_tex} \qquad
  \def\svgwidth{.7\columnwidth}
  \input{img/scheme/fig2.pdf_tex} \qquad
  \def\svgwidth{.6\columnwidth}
  \input{img/scheme/fig3.pdf_tex}
  
  \caption{Schematical representation of the particles scattering geometry. Previous methods, including Lorenz-Mie theory, assume independent scattering of particles (left), assuming that the distance $\tPx_{ij}$ between two particles $i$ and $j$ is very large (i.e., $\tPx_{ij}\rightarrow\infty$), neglecting the potential interactions between particles. In our work (middle) we differentiate between near field scattering of particles within a small region in space (cluster $\Cls$ centered at $\Px_\Cls$), and particles $k$ on the far-field region of the cluster (distance $\tPx_{\Cls k}\rightarrow\infty$). For large values of $\tPx_{\Cls k}$, the direction between particle $k$ and any particle $j\in\Cls$ is $dPx_{ik}\approx\dPx_{\Cls k}$: Therefore, we can assume a planar exciting field $\ExcEField_{\Cls k}(\px)$ on the whole cluster $\Cls$ from particle $k$, with direction $\dPx_{\Cls k}$ (right). }
\label{fig:diagram}
\end{figure*}

\subsection{Foldy-Lax Equations}
\label{ssec:foldy-lax}
%
We now consider a medium filled with $N$ finite discrete particles with volume $V_i$ and index of refraction $\sIOR_i(\px)$. Considering an incident E-field $\IncEField(\px)$, we can rewrite \Eq{eq:vri} as
%
\begin{equation}
    \EField(\px) = \IncEField(\px) + \int_{\Real^3} U(\px')\,\sGreen(\px,\px') \cdot \EField(\px') \diff{\px'},
    \label{eq:EfieldParticles}
\end{equation}
%
where $\sGreen(\px,\px')$ is the dyadic Green's function~\eqref{eq:greenfunc}, and $U(\px)$ the potential function given by
%
\begin{equation}
    U(\px) = \sum_{i=1}^{N} U_i(\px) \quad \text{with} \quad U_i(\px) = \begin{cases} 
    0, & (\px \notin V_i)\\ 
    k_1^2[\sIOR_i^2(\px)-1]. & (\px \in V_i)
    \end{cases}
    \label{eq:potential}
\end{equation}
%
By combining \Eqs{eq:EfieldParticles}{eq:potential}, we can express the field at any position $\px\in\Real^3$ following the so-called \emph{Foldy-Lax equation}~\cite{foldy1945multiple,lax1951multiple} as
%
\begin{equation}
\EField(\px) = \IncEField(\px) + \sum_{i=1}^N \overbrace{\int_{V_i} \sGreen(\px,\px')\cdot \int_{V_i} \dyad{T}_i(\px',\px'') \cdot  \EField_i(\px'') \diff{\px''}\,\diff{\px'}}^{\eqdef \,\ScaEField_i(\px)}    \label{eq:foldylax}
\end{equation}
%
with $\ScaEField_i(\px)$ and $\EField_i(\px)$ the scattered and partial field of particle $i$, and $\dyad{T}_j(\px,\px')$ $\dyad{T}_i(\px,\px')$ the dyad transition operator for particle $i$ defined as~\cite{tsang1985theory} 
\begin{equation}
\label{eq:dyadtransition}
    \begin{split}
        \dyad{T}_i(\px,\px') =\;& U_i(\px) \,\delta(\px-\px')\,\sIdDyad \\
        & + U_i(\px) \int_{V_i} \sGreen(\px,\px'') \cdot \dyad{T}(\px'',\px') \diff{\px''},
    \end{split}
\end{equation}
%
with $\delta(x)$ the Dirac delta. 
%
The partial field at particle $i$ is defined as $\EField_i(\px)=\IncEField(\px) + \sum_{j(\neq i)=1}^N \ExcEField_{ij}(\px)$, where the partial exciting field $\ExcEField_{ij}(\px)$ from particles $j$ to $i$ is 
\begin{equation}
\ExcEField_{ij}(\px) = \int_{V_j} \sGreen(\px,\px')\int_{V_j} \dyad{T}_j(\px',\px'') \EField_j(\px'') \diff{\px''}\,\diff{\px'},
\label{eq:excfield}
\end{equation}
%
with $\px\in V_i$. Note that the scattered and exciting fields for particle $j$ have essentially the same form. 
%
As shown by Mishchenko \shortcite{mishchenko2002vector}, the Foldy-Lax equation~\eqref{eq:foldylax} solves exactly the volume integral equation~\eqref{eq:vri} for multiple arbitrary particles in the medium, without any assumptions on their composition or packing rate, beyond the assumption of a homogeneous hosting medium.

%\sz{Maybe also review the Mie theory (by talking about what it solved under this formulation. Also, need to describe how the key equations are used by us (e.g., what is being solved by our simulations).}

\paragraph{Far-field Foldy-Lax Equations}
\Eq{eq:excfield} defines the exact exciting field resulting from the scattering by particle~$j$ on particle~$i$.
However, if the distance $\tPx_{ij} \defeq \| \Px_i - \Px_j \|$ between particles (with $\Px_i$ denoting the center of particle $i$) is large, we can approximate the propagation distance between any point $\px \in V_i$ and $\px' \in V_j$ as
%
\begin{equation}
    \| \px - \px' \| \approx \tPx_{ij} + (\dPx_{ij} \cdot {\Delta}\px) -  (\dPx_{ij} \cdot {\Delta}\px'),
\end{equation}
%
with $\dPx_{ij} \defeq \nicefrac{(\Px_i - \Px_j)}{\tPx_{ij}}$, ${\Delta}\px \defeq \px - \Px_i$ and ${\Delta}\px' \defeq \px' - \Px_j$ (see Figure~\ref{fig:diagram}, left).
With this approximation, we can now express $\ExcEField_{ij}(\px)$ for a point $\px \in V_i$ using its \emph{far-field} approximation, as:%
\footnote{We note that, accordingly to Mishchenko~\shortcite{mishchenko2006multiple}, the product would require to multiply the integrand by the dyad $(\sIdDyad - \dPx_{ij}\otimes\dPx_{ij})$ to ensure a transverse planar field; we remove it for clarity.}
%
\begin{equation}
    \begin{split}
        & \ExcEField_{ij}(\px)\\
        \approx\;& \frac{\E^{\img k_1 (\tPx_{ij}+\dPx_{ij}\cdot{\Delta}\px)}}{4\pi\tPx_{ij}} \int_{V_j} \sGreenProp(\dPx_{ij},{\Delta}\px') \int_{V_j} \dyad{T}_j(\px',\px'')\cdot \EField_j(\px'') \diff{\px''}\,\diff{\px'} \\
        =\;& \frac{\exp(\img k_1 \,\tPx_{ij})}{\tPx_{ij}} 
        \sGreenProp(\dPx_{ij}, \Delta \px) \,\ExcEField_{1ij}(\dPx_{ij}),
    \end{split}
    \label{eq:excfieldfar}
    \raisetag{17pt}
\end{equation}
%
where: $\px \in V_i$ is a point in particle $i$; $\sGreenProp(\dw, \Delta \px)=\exp(\img k_1 \,\dPx_{ij}\cdot{\Delta}\px)$; and $\ExcEField_{1ij}$ is the far-field exciting field from particle $j$ to particle $i$ that is solely characterized by the propagation direction $\dPx_{ij}$. In order for \Eq{eq:excfieldfar} to be valid, the distance $\tPx_{ij}$ needs to hold the far-field criteria, which relates the $\tPx_{ij}$ with the radius of the particle $\radius_j$ following the inequality~\cite{mishchenko2006multiple}:
%
\begin{equation}
    k_1 \tPx_{ij} \gg \max\left(1, \frac{k_1^2\radius_j^2}{2}\right).
    \label{eq:farfield}
\end{equation}
%
%The Lorenz-Mie theory~\cite{hulst1981light} builds upon this far-field assumption to model electromagnetic scattering from small (spherical) particles which as shown by Mishchenko~\shortcite{mishchenko2002vector} is at the core assumptions of radiative transfer theory. 
This far-field assumption is both the basis for the Lorenz-Mie theory~\cite{hulst1981light} (to model electromagnetic scattering from small spherical particles) and, as shown by Mishchenko~\shortcite{mishchenko2002vector}, at the core of the radiative transfer theory.

In the following, we relax the assumption of near field scattering and compute the Foldy-Lax equations for clusters of particles for both the near- and far-field regions. Then, we use them to compute the scattering matrix to be used in the RTE to efficiently approximate light transport between clusters of particles. 
