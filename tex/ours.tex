\section{Computing the Bulk Scattering Parameters}
\label{sec:ours_numerical}
%
We now detail our numerical computations of the scattering dyad $\ScaDyad_\Cls(\dwi,\dws)$ of \Eq{eq:farscatdyadC}, \rev{which in turn determines the bulk scattering parameters following Equations~\eqref{eq:crosstcluster}--\eqref{eq:sigmascluster}. These bulk scattering parameters}can be directly used in any renderer supporting participating media~\cite{novak2018monte} \rev{using tabulated phase function and cross sections.}

Computing $\ScaDyad_\Cls(\dwi,\dws)$ essentially boils down to solving the time-harmonic Maxwell equations for an incident unit-amplitude planar field with direction $\dwi$. While several different methods exist for that purpose (see \S 16 of~\cite{mishchenko2014electromagnetic} for an overview), we opt for the superposition T-matrix method~\cite{mackowski1996calculation} that has been demonstrated efficient for moderately large $\Ncls$, can handle scatterers with arbitrary geometry, and is based on the principles of the Foldy-Lax equations, making it particularly appealing for our work. 

In practice, we use the open-source CUDA-based \texttt{CELES} solver \cite{egel2017celes}, which implements the superposition T-matrix method proposed by Mackowski and Mishchenko \shortcite{mackowski2011multiple} for spherical or randomly rotated particles.
In our implementation, we focus on clusters of spherical particles.
Since the Lorenz-Mie theory also assumes spherical particles, this allows us to directly compare our results with those computed using the Lorenz-Mie theory \rev{(see Figures~\ref{fig:mie} and \ref{fig:mie2}). Note that the T-matrix method does not introduce assumptions on the size of particles but, similar to Lorenz-Mie theory, larger particles result in more expensive computations. }

To compute the average scattering dyad $\EV{\ScaDyad_\Cls(\dwi,\dws)}$, we average the scattered field of several random realizations of the clusters (each of which obtained by randomly sampling the position of the particles inside the cluster's bounding sphere).
As we will demonstrate in \S\ref{sec:result}, we use a wide array of distributions including particles uniformly distributed over the volume of the cluster, positively-correlated particles following Shaw et al.~\shortcite{shaw2002super}, negatively-correlated particles using Poisson sampling of the sphere, and anisotropic distributions by uniformly sampling the particles on a oriented 2D disk.

Lastly, we represent the resulting phase function as well as the extinction and scattering cross sections as tabulated (i.e., piecewise constant) functions that can be used for rendering.

%\AJ{Maybe add some detail on the tabulation? How many angular bins do we use? How much in the case of anisotropic PFs?How much on the spectral domain?} \sz{Yes, we should provide some details here.}
