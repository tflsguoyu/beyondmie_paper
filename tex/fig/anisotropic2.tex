\begin{figure}[t]
    \centering
    \setlength{\resLen}{1.06in}
    \addtolength{\tabcolsep}{-3pt}
    \small
    \begin{tabular}{ccc}
        \begin{overpic}[width=\resLen]{lucy/color_aniso_x_700nm.jpg}
            \put(2,2){\color{white} \bfseries x, 700nm}
        \end{overpic}
        &
        \begin{overpic}[width=\resLen]{lucy/color_aniso_y_700nm.jpg}
            \put(2,2){\color{white} \bfseries y, 700nm}
        \end{overpic}
        &
        \begin{overpic}[width=\resLen]{lucy/color_aniso_z_700nm.jpg}
            \put(2,2){\color{white} \bfseries z, 700nm}
        \end{overpic} \\
        \begin{overpic}[width=\resLen]{lucy/color_aniso500_x.jpg}
            \put(2,2){\color{white} \bfseries x, multi.}
        \end{overpic}
        &
        \begin{overpic}[width=\resLen]{lucy/color_aniso500_y.jpg}
            \put(2,2){\color{white} \bfseries y, multi.}
        \end{overpic}
        &
        \begin{overpic}[width=\resLen]{lucy/color_aniso500_z.jpg}
            \put(2,2){\color{white} \bfseries z, multi.}
        \end{overpic}

    \end{tabular}
    \caption{\label{fig:aniso2}
        Renderings of homogeneous Lucy models with the same anisotropic medium as in Figure~\ref{fig:aniso1}.
        The medium's orientation---which determines the axis of the disk---is aligned \rev{respectively} with the $x$-, $y$-, and $z$-axis \rev{in the three columns}, \rev{leading to distinctive appearances}.
        \rev{We show single-wavelength ($\lambda = 700\mathrm{nm}$) renderings on the top and multi-spectral ones on the bottom.}
    }
\end{figure}
