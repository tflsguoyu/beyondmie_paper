\begin{figure*}
\centering
  \def\svgwidth{.5\columnwidth}
  \input{tex/fig/diagram_fig1} \qquad
  \def\svgwidth{.7\columnwidth}
  %% Creator: Inkscape 1.0.2 (e86c870879, 2021-01-15, custom), www.inkscape.org
%% PDF/EPS/PS + LaTeX output extension by Johan Engelen, 2010
%% Accompanies image file 'fig2.pdf' (pdf, eps, ps)
%%
%% To include the image in your LaTeX document, write
%%   \input{<filename>.pdf_tex}
%%  instead of
%%   \includegraphics{<filename>.pdf}
%% To scale the image, write
%%   \def\svgwidth{<desired width>}
%%   \input{<filename>.pdf_tex}
%%  instead of
%%   \includegraphics[width=<desired width>]{<filename>.pdf}
%%
%% Images with a different path to the parent latex file can
%% be accessed with the `import' package (which may need to be
%% installed) using
%%   \usepackage{import}
%% in the preamble, and then including the image with
%%   \import{<path to file>}{<filename>.pdf_tex}
%% Alternatively, one can specify
%%   \graphicspath{{<path to file>/}}
%% 
%% For more information, please see info/svg-inkscape on CTAN:
%%   http://tug.ctan.org/tex-archive/info/svg-inkscape
%%
\begingroup%
  \makeatletter%
  \providecommand\color[2][]{%
    \errmessage{(Inkscape) Color is used for the text in Inkscape, but the package 'color.sty' is not loaded}%
    \renewcommand\color[2][]{}%
  }%
  \providecommand\transparent[1]{%
    \errmessage{(Inkscape) Transparency is used (non-zero) for the text in Inkscape, but the package 'transparent.sty' is not loaded}%
    \renewcommand\transparent[1]{}%
  }%
  \providecommand\rotatebox[2]{#2}%
  \newcommand*\fsize{\dimexpr\f@size pt\relax}%
  \newcommand*\lineheight[1]{\fontsize{\fsize}{#1\fsize}\selectfont}%
  \ifx\svgwidth\undefined%
    \setlength{\unitlength}{195.27076845bp}%
    \ifx\svgscale\undefined%
      \relax%
    \else%
      \setlength{\unitlength}{\unitlength * \real{\svgscale}}%
    \fi%
  \else%
    \setlength{\unitlength}{\svgwidth}%
  \fi%
  \global\let\svgwidth\undefined%
  \global\let\svgscale\undefined%
  \makeatother%
  \begin{picture}(1,0.5215872)%
    \lineheight{1}%
    \setlength\tabcolsep{0pt}%
    \put(0,0){\includegraphics[width=\unitlength,page=1]{scheme/fig2.pdf}}%
    \put(0.92645221,0.21560272){\makebox(0,0)[lt]{\lineheight{1.25}\smash{\begin{tabular}[t]{l}$\Px_k$\end{tabular}}}}%
    \put(0.56274836,0.21221995){\makebox(0,0)[lt]{\lineheight{1.25}\smash{\begin{tabular}[t]{l}$\dPx_{\Cls k}$\end{tabular}}}}%
    \put(0.33172694,0.34184498){\makebox(0,0)[lt]{\lineheight{1.25}\smash{\begin{tabular}[t]{l}$\Px_i$\end{tabular}}}}%
    \put(0.15998663,0.24282036){\makebox(0,0)[lt]{\lineheight{1.25}\smash{\begin{tabular}[t]{l}$\Px_\Cls$\end{tabular}}}}%
    \put(0,0){\includegraphics[width=\unitlength,page=2]{scheme/fig2.pdf}}%
    \put(0.47642379,0.08204267){\makebox(0,0)[lt]{\lineheight{1.25}\smash{\begin{tabular}[t]{l}$\tPx_{\Cls k}$\end{tabular}}}}%
    \put(0,0){\includegraphics[width=\unitlength,page=3]{scheme/fig2.pdf}}%
    \put(0.59047783,0.30130627){\makebox(0,0)[lt]{\lineheight{1.25}\smash{\begin{tabular}[t]{l}$\dPx_{ik}$\end{tabular}}}}%
    \put(0.07747599,0.39359056){\makebox(0,0)[lt]{\lineheight{1.25}\smash{\begin{tabular}[t]{l}$\Cls$\end{tabular}}}}%
    \put(0,0){\includegraphics[width=\unitlength,page=4]{scheme/fig2.pdf}}%
  \end{picture}%
\endgroup%
 \qquad
  \def\svgwidth{.6\columnwidth}
  \input{tex/fig/diagram_fig3}
  
  \caption{Schematical representation of the particles scattering geometry. Previous methods, including Lorenz-Mie theory, assume independent scattering of particles (left), assuming that the distance $\tPx_{ij}$ between two particles $i$ and $j$ is very large (i.e. $\tPx_{ij}\rightarrow\infty$), neglecting the potential interactions between particles. In our work (middle) we differentiate between near field scattering of particles within a small region in space (cluster $\Cls$ centered at $\Px_\Cls$), and particles $k$ on the far-field region of the cluster (distance $\tPx_{\Cls k}\rightarrow\infty$). For large values of $\tPx_{\Cls k}$, the direction between particle $k$ and any particle $j\in\Cls$ is $dPx_{ik}\approx\dPx_{\Cls k}$: Therefore, we can assume an planar exciting field $\ExcEField(\px)_{\Cls k}$ on the whole cluster $\Cls$ from particle $k$, with direction $\dPx_{\Cls k}$ (right). }
\label{fig:diagram}
\end{figure*}