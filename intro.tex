\section{Introduction}
\label{sec:intro}
%
Participating media and translucent materials---such as marble, milk, wax, and human skin---are ubiquitous in the real world. These materials allow light to penetrate their surfaces and scatter in the interior. 
%
In computational optics and computer graphics, how light interacts with participating media and translucent materials is typically modeled using the radiative transfer theory (RTT). Under this formulation, a participating medium consists of microscopic particles (\emph{scatterers}) randomly dispersed in some homogeneous embedding medium. After entering a translucent material, light travels in straight lines in the embedding medium and occasionally collides with a particle and gets redirected into a new direction. To capture the macroscopic behavior of light, the RTT uses a statistical description of the particles (the medium bulk parameters), namely the extinction coefficient $\sigT$ (aka. optical density), the scattering coefficient $\sigS$, and the phase function $\phase$.

While purely phenomenological in origin, the RTT has been demonstrated a corollary of Maxwell equations, under the assumption of far-field or independent scattering~\cite{mishchenko2002vector}. Therefore, these optical bulk parameters can be obtained from first principles, using e.g. Lorenz-Mie theory~\cite{hulst1981light,frisvad2007computing}. However, although very successful in practice, this theory neglects the interactions occurring between particles in their near-field, including wave-optics effects such as diffraction and interference with neighbor particles. Consequently, Lorenz-Mie theory is largely limited to isotropic media with relatively low packing rates.


Previously, the classical radiative transfer theory has been generalized to handle materials with (statistically) organized microstructures. 
Anisotropic media~\cite{jakob2010radiative}, for instance, have bulk scattering parameters with stronger directional dependency compared to isotropic media.
Additionally, media comprised of particles with correlated locations can exhibit non-exponential transmittance and characteristic scattering profiles~\cite{bitterli2018radiative,jarabo2018radiative}.
Although several empirical models have been proposed to model these media, these still base on the very same far-field assumption of Lorenz-Mie scattering. Therefore, techniques capable of computing the bulk optical parameters of a material, based its microscopic properties, have been lacking.

In this paper, we bridge this gap by introducing a new technique to systematically and rigorously compute the bulk scattering parameters. %for media formed by clusters of particles in their near-field region.
The elementary building block of our technique is \emph{particle clusters} in which individual particles follow user-specified distributions.
Within a cluster, we consider full near-field light transport effects; Between clusters, on the contrary, we use a far-field approximation to allow efficient modeling of macroscopic level light transport.

Our formulation is derived from first principles of light transport (i.e., Maxwell electromagnetism) and reduces to the Lorenz-Mie theory in the special case of single-particle scatterers. Based on this formulation, we demonstrate how the bulk parameters can be computed numerically. Using our technique, we systematically generate radiative transfer optical parameters capturing multi-spectral, anisotropic, and correlated scattering effects for particles with arbitrary distributions (Figure~\ref{fig:teaser}).

Concretely, our contributions include:
%
\begin{itemize}
    \item Establishing a computational framework for modeling light scattering from clusters of particles (\S\ref{sec:ours_theory}).
    %
    \item Showing how radiatve transfer parameters can be computed numerically based on our formulation (\S\ref{sec:ours_numerical}).
    %
    \item Demonstrating how our technique can be applied to systematically compute scattering parameters for a variety of participating media (\S\ref{sec:result}).
\end{itemize}
