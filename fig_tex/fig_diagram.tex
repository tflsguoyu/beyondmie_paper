\begin{figure*}
\centering
  \def\svgwidth{.5\columnwidth}
  \input{images/scheme/fig1.pdf_tex} \qquad
  \def\svgwidth{.7\columnwidth}
  \input{images/scheme/fig2.pdf_tex} \qquad
  \def\svgwidth{.6\columnwidth}
  \input{images/scheme/fig3.pdf_tex}
  
  \caption{Schematical representation of the particles scattering geometry. Previous methods, including Lorenz-Mie theory, assume independent scattering of particles (left), assuming that the distance $\tPx_{ij}$ between two particles $i$ and $j$ is very large (i.e., $\tPx_{ij}\rightarrow\infty$), neglecting the potential interactions between particles. In our work (middle) we differentiate between near field scattering of particles within a small region in space (cluster $\Cls$ centered at $\Px_\Cls$), and particles $k$ on the far-field region of the cluster (distance $\tPx_{\Cls k}\rightarrow\infty$). For large values of $\tPx_{\Cls k}$, the direction between particle $k$ and any particle $j\in\Cls$ is $dPx_{ik}\approx\dPx_{\Cls k}$: Therefore, we can assume a planar exciting field $\ExcEField_{\Cls k}(\px)$ on the whole cluster $\Cls$ from particle $k$, with direction $\dPx_{\Cls k}$ (right). }
\label{fig:diagram}
\end{figure*}