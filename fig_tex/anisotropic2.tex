\begin{figure}[t]
    \centering
    \setlength{\resLen}{1.06in}
    \addtolength{\tabcolsep}{-3pt}
    \small
    \begin{tabular}{ccc}
        \begin{overpic}[width=\resLen]{images/lucy/aniso_x.jpg}
            \put(2,2){\color{white} \bfseries x}
        \end{overpic}
        &
        \begin{overpic}[width=\resLen]{images/lucy/aniso_y.jpg}
            \put(2,2){\color{white} \bfseries y}
        \end{overpic}
        &
        \begin{overpic}[width=\resLen]{images/lucy/aniso_z.jpg}
            \put(2,2){\color{white} \bfseries z}
        \end{overpic}
    \end{tabular}
    \caption{\label{fig:aniso2}
        Renderings of homogeneous Lucy models with the same anisotropic medium as in Figure~\ref{fig:aniso1}.
        With the medium's orientation---which determines the axis of the disk---aligned with the $x$-, $y$-, and $z$-axis, respectively, the Lucy model exhibit distinct appearances.
    }
\end{figure}
