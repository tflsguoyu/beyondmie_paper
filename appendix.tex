\section*{\Huge{\textbf{Appendix}}}

In this document we derive the far-field scattered field of clusters of particles from the Foldy-Lax equations. \rev{For completeness and selfcontainedness, we start by reviewing time-harmonic electromagnetics following the derivations described by Mishchenko et al.~\shortcite{mishchenko2006multiple} (\Sec{sec:vri}), and its formulation for a medium with multiple particles embedded using the Foldy-Lax equations~\cite{foldy1945multiple,lax1951multiple}(\Sec{sec:foldylax}) and their far-field approximations (\Sec{sec:fffoldylax}.}


From these, we later derive the scattering dyad encoding the response of a cluster of particles in the far field, which later can be used to compute the (radiative) optical properties of a scattering medium. 

\section{Electromagnetic Scattering}
\label{sec:vri}
The propagation of a time-harmonic monochromatic electromagnetic field with frequency $\sFreq$ is defined by the Maxwell curl equations as
\begin{align}
\nabla\times \EField(\px) = \img \sFreq \sPermeability(\px) \MField(\px), \nonumber \\
\nabla\times \MField(\px) = -\img \sFreq \sPermittivity(\px) \EField(\px),
\label{eq:maxwell}
\end{align}
%
with $\nabla\times .$ the curl operator, $\EField(\px)$ and $\MField(\px)$ the electric and magnetic field at $\px$ respectively, $\sPermeability(\px)$ and $\sPermittivity(\px)$ the magnetic permeability and electric permittivity at $\px$ respectively, and $\img=\sqrt{-1}$.

\rev{By assuming a non-magnetic medium (i.e. $\sPermeability(\px)=\sPermeability_0$, with $\sPermeability_0$ the magnetic permeability of a vacuum), and taking the curl on the first line in \Eq{eq:maxwell} we get
\begin{align}
\nabla^2 \EField(\px) &= \img \sFreq \sPermeability(\px) \nabla\times\MField(\px)\nonumber\\
&=-\img^2 \sFreq^2 \sPermeability_0 \sPermittivity(\px) \EField(\px),
\label{eq:maxwellcurl}
\end{align}
%
with $\nabla^2=\nabla\times\nabla$, which by arithmetic reordering reduces to the \emph{electric field wave equation}
}
\begin{equation}
\nabla^2\times\EField(\px) - k(\px)^2\EField(\px) = 0,
\label{eq:efieldwave}
\end{equation}
where $k(\px)=\sFreq\sqrt{\sPermittivity(\px)\sPermeability_0}$ is the medium's wave number at $\px$. \rev{Note that the wave number $k$ has a dependence on the frequency $\sFreq$; in the following we omit such dependence for brevity.}

\rev{Let us now assume an infinite homogeneous isotropic medium with permittivity $\sPermittivity_1$, filled with scatterers with potentially inhomogeneous permittivity $\sPermittivity_2(\px)$. This separates the space in two different regions: The surrounding infinite region $V_0$, and the finite disjoint region occupìed by the scatterers $V$, so that $V \bigcup V_0 = \Real^3$.
%
Under this configuration, we can express \Eq{eq:efieldwave} as two different wave equations
\begin{align}
\label{eq:efieldmedium}
 &\nabla^2\times\EField(\px) - k_1^2\EField(\px) = 0, \px \in V_0, \\ 
 \label{eq:efieldscat}
 &\nabla^2\times\EField(\px) - k_2(\px)^2\EField(\px) = 0, \px \in V,
\end{align}
%
with $k_1$ the constant wave number at the hosting medium, and $k_2(\px)$ the potentially inhomogeneous wave number at the scatterers. \Eqs{eq:efieldmedium}{eq:efieldscat} can be expressed together in a single inhomogeneous differential equation as
\begin{equation}
\nabla^2\times\EField(\px) - k_1^2\EField(\px) =  U(\px)\,\EField(\px), 
\label{eq:efieldboth}
\end{equation}
with $U(\px)=k_1^2 [\sIOR^2(\px)-1]$ the potential function at $\px$, and $\sIOR(\px)=k_2(\px)/k_1$ the index of refraction at $\px$. It is trivial to verify that for $\px\in V_0$ in the hosting medium $\sIOR(\px)=1$, then the potential function $U(\px)$ vanishes, and \Eq{eq:efieldboth} reduces to \Eq{eq:efieldmedium}. 

Solving the inhomogeneous linear differential equation described in \Eq{eq:efieldboth} results in two terms: The contribution of the incident field $\IncEField(\px)$, which is the sole contribution in the case of a homogeneous medium, and the scattered field $\ScaEField(\px)$ resulting of introducing inhomogeneities (i.e. scatterers) in the embedding medium, as
\begin{equation}
\EField(\px) = \IncEField(\px) + \ScaEField(\px).
\label{eq:vrisum}
\end{equation}

The fist part trivially satisfies \Eq{eq:efieldmedium} for the incident field $\IncEField(\px)$. In order to compute the scattered field $\ScaEField(\px)$, we enforce energy conservation by computing a solution that vanishes at large distances. We introduce the free-space dyadic Green function $\sGreen(\px,\px')$ that satisfies the impulse response of the linear system in \Eq{eq:efieldwave}, modeled as 
\begin{equation}
 \nabla^2\times\sGreen(\px,\px') - k_1^2\,\sGreen(\px,\px') = \sIdDyad\,\delta(\px-\px'),
 \label{eq:diffgreen}
 \end{equation}
where $\sIdDyad$ is the identity dyad, and $\delta(\cdot)$ is the Dirac delta function. Note that the derivatives are with respect to $\px$. Multiplying both sides of the differential equation by $U(\px)\,\EField(\px)$, and integrating both sides with respect to $\px'$ over the entire space $\Real^3$, we get
\begin{align}
\left( \nabla^2\times\sIdDyad - k_1^2\,\sIdDyad\, \right)\overbrace{\int_{\Real^3} U(\px')\, \sGreen(\px,\px') \cdot \EField(\px')  \diff{\px'}}^{\ScaEField(\px)} = U(\px)\,\ScaEField(\px),
\end{align}
%
with $.\cdot.$ the dyadic-vector dot-product. 
Since the potential function $U(\px)$ vanishes everywhere outside $V$, we can express the scattered field $\ScaEField(\px)$ as an integral on the space occupied by scatterers $V$ only, as
\begin{align}
\ScaEField(\px) & = \int_V U(\px)  \sGreen(\px,\px') \cdot \EField(\px') \diff{\px'} \nonumber \\
& = k_1^2\,\int_V [\sIOR^2(\px')-1] º\, \sGreen(\px,\px') \cdot \EField(\px') \diff{\px'}.
\label{eq:vriscatt}
\end{align}

Now, the only term missing for computing $\ScaEField(\px)$ is the Green function that solves \Eq{eq:diffgreen}, which has a well-known solution as
\begin{equation}
\sGreen(\px,\px') = \left(\sIdDyad+k_1^{-2}\nabla\otimes\nabla\right) \frac{\exp(\img k_1 |\px-\px'|)}{4\pi|\px-\px'|},
\label{eq:greenfunc}
\end{equation}
where $. \otimes .$ denotes the dyadic product of two vectors, and the derivative operator $\nabla$ applies over $\px$.

Finally, by plugin \Eq{eq:vriscatt} into \Eq{eq:vrisum} we get the \emph{volume integral equation}~\cite[Sec.3.1]{mishchenko2006multiple} that solves the Maxwell equations~\eqref{eq:maxwell} as the sum of the incident field $\IncEField(\px)$ and the scattered field $\ScaEField(\px)$ due to inhomogeneities in the medium in the form of scatterers:
}
\begin{align}
\EField(\px) & = \IncEField(\px)+\ScaEField(\px) \nonumber\\
&=\IncEField(\px) + \int_V \underbrace{k_1^2 [\sIOR^2(\px')-1]}_{U(\px')} \sGreen(\px,\px') \cdot \EField(\px') \diff{\px'}.
\label{eq:vri}
\end{align}
Intuitively, \Eq{eq:vri} models the scattering field as the superposition of the spherical wavelets resulting from a change of permitivitty (i.e. with $\sIOR(\px')\neq1$). This is a general equation that solves the Maxwell equations for non-magnetic media in arbitrary setups. Note also the recursive nature of \Eq{eq:vri}; we will deal with this recursivity in the following section, computing $\ScaEField(\px)$ as a function of the incident field $\IncEField(\px)$. 


%%%%%%%%%%%%%%%%%%%%%%%%%%%%%%%%%%%%%%%%%%%%%%%%%%%%%%%%%%%%%%%%%%%%%%
%%%%%%%%%%%%%%%%%%%%%%%%%%%%%%%%%%%%%%%%%%%%%%%%%%%%%%%%%%%%%%%%%%%%%%
%%%%%%%%%%%%%%%%%%%%%%%%%%%%%%%%%%%%%%%%%%%%%%%%%%%%%%%%%%%%%%%%%%%%%%
\section{Foldy-Lax Equations}
\label{sec:foldylax}
\rev{Let us consider a medium filled with $N$ finite discrete particles with volume $V_i$ and index of refraction $\sIOR_i(\px)$. 
%
We can now define the potential function $U_i(\px)$ for each particle $i$ as
\begin{equation}
U_i(\px) = \left\{\begin{matrix} 
0, & \px\not\in V_i\\ 
k_1^2[\sIOR_i^2(\px)-1] & \px \in V_i, \end{matrix} \right.
\label{eq:potential}
\end{equation}
%
with the total potential function $U$ in \Eq{eq:vri} defined as $U(\px) = \sum_{i=1}^{N} U_i(\px)$.}
%
By combining \Eqs{eq:vri}{eq:potential}, we can express the field at any position $\px\in\Real^3$ following the so-called \emph{Foldy-Lax equation}~\cite{foldy1945multiple,lax1951multiple} as
\begin{align}
\EField(\px)& = \IncEField(\px) + \sum_{i=1}^N \int_{V_i} \sGreen(\px,\px') \cdot \int_{V_i} \dyad{T}_i(\px',\px'') \cdot \EField_i(\px'') \diff{\px''}\,\diff{\px'} \nonumber \\
&=\IncEField(\px)+\sum_{i=1}^N \ScaEField_i(\px),
\label{eq:foldylax}
\end{align}
%
with $\EField_i(\px)=\IncEField(\px) + \sum_{j(\neq i)=1}^N \ExcEField_{ij}(\px)$, where the partial exciting field $\ExcEField_{ij}(\px)$ from particles $j$ to $i$ and $\ScaEField_i(\px)$ the scattered field from particle $i$. \rev{Note that we overload the dot-product operator accounting for the dyad-dyad case.}The dyad transition operator $\dyad{T}_i(\px,\px')$ for particle $i$ defined as~\cite{tsang1985theory}
\begin{align}
\dyad{T}_i(\px,\px') & = U_i(\px) \delta(\px-\px')\sIdDyad   \nonumber
\\
& +U_i(\px) \int_{V_i} \sGreen(\px,\px'') \cdot \dyad{T}_i(\px'',\px') \diff{\px''},
\end{align}
with $\delta(x)$ the Dirac delta, $\sIdDyad$ the identity dyad. The partial exciting field $\ExcEField_{ij}(\px)$ is defined as
\begin{equation}
\ExcEField_{ij}(\px) = \int_{V_j} \sGreen(\px,\px') \cdot \int_{V_j} \dyad{T}_j(\px',\px'') \cdot \EField_j(\px'') \diff{\px''}\,\diff{\px'}, 
\label{eq:excfield}
\end{equation}
%
with $\px\in V_i$. Note that the exciting field $\ExcEField_{ij}(\px)$ has essentially the same form as the scattered field $\ScaEField_j(\px)$ from particle $j$. As shown by Mishchenko~\shortcite{mishchenko2002vector}, the Foldy-Lax equations~\eqref{eq:foldylax} solve exactly the volume integral equation~\eqref{eq:vri} for multiple arbitrary particles in the medium without any assumptions on their composition or packing rate, beyond the assumption of a homogeneous hosting medium. 





%%%%%%%%%%%%%%%%%%%%%%%%%%%%%%%%%%%%%%%%%%%%%%%%%%%%%%%%%%%%%%%%%%%%%%
%%%%%%%%%%%%%%%%%%%%%%%%%%%%%%%%%%%%%%%%%%%%%%%%%%%%%%%%%%%%%%%%%%%%%%
%%%%%%%%%%%%%%%%%%%%%%%%%%%%%%%%%%%%%%%%%%%%%%%%%%%%%%%%%%%%%%%%%%%%%%
\section{Far-Field Foldy-Lax Equations}
\label{sec:fffoldylax}
\Eq{eq:excfield} define the exact exciting field resulting from scattering by particle $j$ on particle $i$. 
%
However, if the distance between particles $\tPx_{ij}=|\Px_i-\Px_j|$, with $\Px_i$ the origin of particle $i$, is large, \rev{so that $k_1 \tPx_{ij} \gg 1$,}we can approximate the propagation distance between points $\px\in V_i$ and $\px'\in V_j$ as $|\px-\px'|\approx \tPx_{ij} + (\dPx_{ij}\cdot {\Delta}\px) -  (\dPx_{ij}\cdot {\Delta}\px')$, with $\dPx_{ij}=\frac{\Px_i-\Px_j}{\tPx_{ij}}$, ${\Delta}\px=\px-\Px_i$ and ${\Delta}\px'=\px'-\Px_j$. \rev{With this approximation, and after some algebraic operations, we can now approximate the dyadic Green's function as
\begin{equation}
\sGreen(\px,\px') \approx (\sIdDyad - \dPx_{ij}\otimes\dPx_{ij}) \frac{\exp(\img k_1 \tPx_{ij})}{4\pi\tPx_{ij}} \sGreenProp(\dPx_{ij}, \Delta \px) \sGreenProp(-\dPx_{ij},\Delta\px'),
\end{equation}
%
with $\sGreenProp(\dw,\px)=\exp(\img k_1 \dw\cdot \px)$. 
With this approximation, we can now express $\ExcEField_{ij}(\px)$ for a point $\px\in V_i$ using its \emph{far-field} approximation, as%\footnote{Note that accordingly to Mishchenko~\shortcite{mishchenko2002vector} the product would require to multiply the integrand by the dyad $(\sIdDyad - \dPx_{ij}\otimes\dPx_{ij})$ to ensure a transverse planar field; we remove it for clarity.}:
%
\begin{equation}
\ExcEField_{ij}(\px) = \frac{\exp(\img k_1 \,\tPx_{ij})}{\tPx_{ij}} \sGreenProp(\dPx_{ij}, \Delta \px) \,\ExcEField_{1ij}(\dPx_{ij}),
\label{eq:ffgreen}
\end{equation}
%
with $\px\in V_i$ a point in particle $i$, and $\ExcEField_{1ij}$ the far-field exciting field from particle $j$ to particle $i$ defined as
%
\begin{equation}
    \ExcEField_{1ij}(\dPx_{ij}) = \frac{(\sIdDyad - \dPx_{ij}\otimes\dPx_{ij})}{4\pi} \cdot \int_{V_j} g(-\dPx_{ij}, {\Delta}\px') \int_{V_j} \dyad{T}_j(\px',\px'') \cdot \EField_j(\px'') \diff{\px''}\diff{\px'}.
    \label{eq:excfieldfar}
\end{equation} 
The dyad $(\sIdDyad - \dPx_{ij}\otimes\dPx_{ij})$ to ensure a transverse planar field, which allows to solely characterize $\ExcEField_{1ij}(\dPx_{ij})$ by the propagation direction $\dPx_{ij}$.}In order to \Eq{eq:excfieldfar} to be valid, the distance $\tPx_{ij}$ needs to hold the far-field criteria, which relates the $\tPx_{ij}$ with the radius of the particle $\radius_j$ following the inequality~\cite{mishchenko2006multiple}
\begin{equation}
k_1 \tPx_{ij} \gg \max\left(1, \frac{k_1^2\radius_j^2}{2}\right).
\label{eq:farfield}
\end{equation}

The two forms of computing the exciting field from particle $j$ to $i$ (\Eqs{eq:excfield}{eq:excfieldfar}) suggest that we can consider two subsets of particles $j$ depending on their distance with respect to the point of interest $\px$: One set of $\Nnear$ particles in the near field and another set of $\Nfar$ particles in the far field. With that, we can now the exciting field in particle $i$ as
%
\begin{align}
\EField_i(\px)= \IncEField(\px) + \sum_{j(\neq i)=1}^\Nnear \ExcEField_{ij}(\px) + \sum_{k=1}^\Nfar \ExcEField_{ik}(\px).
\label{eq:foldylaxtwo}
\end{align}
%
%
In the following, we will use this as motivation for defining the exciting field on a particle from a group of particles in the far field. 


%%%%%%%%%%%%%%%%%%%%%%%%%%%%%%%%%%%%%%%%%%%%%%%%%%%%%%%%%%%%%%%%%%%%%%
%%%%%%%%%%%%%%%%%%%%%%%%%%%%%%%%%%%%%%%%%%%%%%%%%%%%%%%%%%%%%%%%%%%%%%
%%%%%%%%%%%%%%%%%%%%%%%%%%%%%%%%%%%%%%%%%%%%%%%%%%%%%%%%%%%%%%%%%%%%%%
\section{Far-Field Foldy-Lax Equations for Clusters of Particles}
\label{sec:farfield_foldy_clusters}
Here we derive the far-field Foldy-Lax equations for groups of particles where the a cluster of these particles are in their respective near-field region, while the other elements in the system are in the far field. For simplicity in our derivations, we consider a single far-field incident field, as well as single particle $k$ in the far field region of the cluster of particles. 
%
More formally, let us now consider a cluster $\Cls$ of $N_\Cls$ particles, where all particles $j\in\Cls$ are in their respective near-field region, and that the particles of the cluster are bounded on a sphere centered at $\Px_\Cls$ with radius $\radius_\Cls$. 

Since both the incident field $\IncEField(\px)$ and the exciting field $\ExcEField_{\Cls k}$ from particle $k$ are in the far-field region, we can assume that both fields are planar waves defined as
\begin{align}
    \label{eq:farincfieldcluster}
    \IncEField(\px) &= \IncEField_0 \,\exp(\img k_1 \dw \cdot \Delta \px) = \IncEField_0\,\sGreenProp(\dw, \Delta \px) , \\
    \ExcEField_{\Cls k}(\px) &= \ExcEField_{0\Cls k}\,\exp(\img k_1 \dPx_{\Cls k} \cdot \Delta \px) =  \ExcEField_{0\Cls k}\,\sGreenProp(\dPx_{\Cls k}, \Delta \px) 
    \label{eq:farexcfieldcluster} 
\end{align}
%
with $\IncEField_0$ and $\ExcEField_{0\Cls k}=\frac{\exp(\img k_1 \,\tPx_{\Cls k})}{\tPx_{\Cls k}}\,\ExcEField_{1\Cls k}(\dPx_{\Cls k})$  \eqref{eq:excfieldfar} the amplitude of the planar incident field and the exciting field from particle $k$ respectively, $\dw$ and $\dPx_{\Cls k}$ the propagation direction of the each field, and $\Delta \px=\px-\Px_\Cls$.
%

Now, let us slightly abuse the dot product notation defining $(\varphi_1 \bigcdot{x} \varphi_2) = \int \varphi_1(x)\cdot\varphi_2(x) \diff x$, and remove the spatial dependency on each term. By the planar incident field assumption, and plugging \Eq{eq:foldylaxtwo} into the definition of the scattered field from particle $i\in\Cls$~\eqref{eq:foldylax}, we get
%
\begin{align}
\label{eq:scafield_cluster1}
\ScaEField_i(\px) & = \sGreen \bigcdot{\px} \dyad{T}_i \bigcdot{\px'} \EField_i \\ & = \sGreen \bigcdot{\px'} \dyad{T}_i \bigcdot{\px''}  \left[\IncEField + \sum_{k=1}^\Nfar \ExcEField_{\Cls k}+ \sum_{j(\neq i)=1}^\Nnear \ExcEField_{ij} \right] \nonumber.
\end{align}
% 
By recursively expanding $\ExcEField_{ij}$, \Eq{eq:scafield_cluster1} becomes
\begin{align}
\label{eq:scafieldcluster3}
\ScaEField_i(\px) = \sGreen \bigcdot{\px'} \dyad{T}_i \bigcdot{\px''} \Bigg[ &\IncEField + \sum_{k=1}^\Nfar \ExcEField_{\Cls k}  \\ 
&  + \sum_{j(\neq i)=1}^\Nnear \sGreen \bigcdot{\px'''} \dyad{T}_j \bigcdot{{\px^{iv}}} \Big[\IncEField + \sum_{k=1}^\Nfar \ExcEField_{\Cls k} + \sum_{l(\neq j)=1}^\Nnear \left[...\right]_l \Big] \Bigg] \nonumber,
\end{align}
%
where the "$[...]_l$" term represents the recursivity as 
\begin{equation}
[...]_l= \sGreen \bigcdot{\px'} \dyad{T}_l \bigcdot{\px''} \Big[\IncEField + \sum_{k=1}^\Nfar \ExcEField_{\Cls k} + \sum_{m(\neq l)=1}^\Nnear \left[...\right]_m\Big] \,.
\label{eq:recursivity1}
\end{equation}
%
By reordering \Eq{eq:scafieldcluster3} we get
%
\begin{align}
\label{eq:scafieldcluster4}
\ScaEField_i(\px) &= \sGreen \bigcdot{\px'} \dyad{T}_i \bigcdot{\px''} \Bigg[ \IncEField + \sum_{j(\neq i)=1}^\Nnear \sGreen \bigcdot{\px'''} \dyad{T}_j \bigcdot{{\px^{iv}}}\Big[\IncEField + \sum_{l(\neq j)=1}^\Nnear \left[...\right]_l^{\IncEField} \Big] \Bigg] \\
& + \sum_{k=1}^\Nfar \left[\sGreen \bigcdot{\px'} \dyad{T}_i \bigcdot{\px''} \Bigg[ \ExcEField_{\Cls k} + \sum_{j(\neq i)=1}^\Nnear \sGreen \bigcdot{\px'''} \dyad{T}_j \bigcdot{{\px^{iv}}}\Big[\ExcEField_{\Cls k} + \sum_{l(\neq j)=1}^\Nnear \left[...\right]_l^{\ExcEField_{\Cls k}} \Big] \Bigg]\right], \nonumber
\end{align}
%
where "$[...]_l^\varphi$" is similar to \Eq{eq:recursivity1}, with form
\begin{equation}
\label{eq:recursivity2}
[...]_l^\varphi= \dyad{T}_l\bigcdot{\px'}\sGreen\bigcdot{\px''} \Big[\varphi + \sum_{m(\neq l)=1}^\Nnear \left[...\right]_m^\varphi\Big] \,.
\end{equation}
%
Finally, by exploiting \Eqs{eq:farincfieldcluster}{eq:farexcfieldcluster}, and contracting the recursion, we transform \Eq{eq:scafieldcluster4} into
%
\begin{align}
\label{eq:scafieldcluster5}
\ScaEField_i(\px) &= \, \sGreen \bigcdot{\px'} \dyad{T}_i \bigcdot{\px''} \Bigg[ \sGreenProp(\dw) + \sum_{j(\neq i)=1}^\Nnear \left[...\right]_j^{\sGreenProp(\dw)} \Bigg] \cdot \IncEField_0\\
& + \sum_{k=1}^\Nfar \left[ \sGreen \bigcdot{\px'} \dyad{T}_i \bigcdot{\px''} \Bigg[ \sGreenProp(\dPx_{\Cls k}) + \sum_{j(\neq i)=1}^\Nnear \left[...\right]_j^{\sGreenProp(\dPx_{\Cls k})} \Bigg]\cdot\ExcEField_{0\Cls k}\right]. \nonumber 
\end{align}
%
Note that each element in the sum in the equation above is the result of the amplitude of the far-field incident or exciting fields, and a series that encode all the near-field scattering in the cluster $\Cls$. We can thus define the scattering dyad $\ScaDyad_i^\text{near}(\dwi,\px)$ relating a field incoming at particle $i$ from direction $\dwi$ with the field at point $\px$ as
\begin{equation}
\boxed{\ScaDyad_i^\text{near}(\dwi,\px) = \sGreen \bigcdot{\px'} \dyad{T}_i\bigcdot{\px''} \Bigg[ \sGreenProp(\dwi) + \sum_{j(\neq i)=1}^\Nnear \left[...\right]_j^{\sGreenProp(\dwi)} \Bigg].}
\label{eq:scatdyad_near}
\end{equation}
%
%

Trivially, following our assumption of constant $\IncEField_0$ and $\ExcEField_{0\Cls k}$ for the whole cluster $\Cls$, we can compute the cluster's scattering dyad as:
\begin{equation}
\ScaDyad_\Cls^\text{near}(\dwi,\px) = \sum_{i=1}^{N_\Cls} \ScaDyad_i^\text{near}(\dwi,\px).
\label{eq:scatdyadcluster_near}
\end{equation}
The scattering dyad $\ScaDyad_\Cls^\text{near}(\dwi,\px)$ solves the scattering field for a unit-amplitude incoming planar field in a scene consisting of the particles forming cluster $\Cls$, and can be computed using any method from computational electromagnetics. 


\paragraph{Far-field approximation}

\Eq{eq:scatdyad_near} represents the general form of the scattering dyad for particle $i$, which results into a five-dimensional function. Assuming that $\px$ is in the far-field region of a particle $i\in\Cls$, \rev{ and by using the far-field approximation of the Green's function~\eqref{eq:ffgreen}, \Eq{eq:scafield_cluster1} becomes
\begin{equation}
\label{eq:scafield_cluster_far}
\ScaEField_i(\px) \approx (\sIdDyad - \dPx_{i}\otimes\dPx_{i}) \frac{\exp(\img k_1 \tPx_i)}{4\pi\tPx_{i}}\cdot \sGreenProp(-\dPx_i) \bigcdot{\px'} \dyad{T}_i \cdot \EField_i,
\end{equation}
%
with $\tPx_i=|\px-\Px_i|$ and $\dPx_i=\frac{\px-\Px_i}{\tPx_i}$. Note that the term $\sGreenProp(\dPx_i,\Delta\px)$ in\Eq{eq:ffgreen} vanishes for a single particle, since $|\Delta\px| = 0$ and therefore $\sGreenProp(\dPx_i,\Delta\px) = 1$.

Now, using the definition of the scattered field $\EField_i$ in \Eq{eq:foldylaxtwo}, and expanding $\ExcEField$ following \Eq{eq:scafield_cluster1}, and expanding $\ExcEField_{ij}$ following \Eq{eq:scafieldcluster3} we get
\begin{align}
\label{eq:scafield_cluster_far2}
\ScaEField_i(\px) &= (\sIdDyad - \dPx_{i}\otimes\dPx_{i}) \frac{\exp(\img k_1 \tPx_i)}{4\pi\tPx_{i}}\cdot \sGreenProp(-\dPx_i) \bigcdot{\px'} \dyad{T}_i \bigcdot{\px''} \Bigg[\IncEField + \sum_{k=1}^\Nfar \ExcEField_{\Cls k}  \nonumber \\ 
&  + \sum_{j(\neq i)=1}^\Nnear \sGreen \bigcdot{\px'''} \dyad{T}_j \bigcdot{{\px^{iv}}} \Big[\IncEField + \sum_{k=1}^\Nfar \ExcEField_{\Cls k} + \sum_{l(\neq j)=1}^\Nnear \left[...\right]_l \Big] \Bigg],
\end{align}
%
with "$[...]_l$" representing the recursivity~\eqref{eq:recursivity1}. Following \Eqs{eq:scafieldcluster4}{eq:scafieldcluster5} we reorder the equation to separate the contribution of the incident $\IncEField$ and exciting fields $\ExcEField_{\Cls k}$ respectively, and exploit the far field assumption to put $\IncEField$ and $\ExcEField_{\Cls k}$ in their planar field form [\Eqs{eq:farincfieldcluster}{eq:farexcfieldcluster}], as
}
\begin{align}
\label{eq:scafield_cluster_far3}
\ScaEField_i(\px) \approx \frac{e^{\img k_1 \tPx_i}}{\tPx_i} \Big(\ScaDyad_i(\dw,\dPx_i)\cdot\IncEField_0
+ \sum_{k=1}^\Nfar \ScaDyad_i(\dPx_{\Cls k},\dPx_i)\cdot\ExcEField_{0\Cls k} \Big),
\end{align}
with $\ScaDyad_i(\dwi,\dws)$ the far-field scattering dyad relating incident and scattered directions $\dwi$ and $\dws$ as
\rev{\begin{empheq}[box=\fbox]{align}
\label{eq:scatdyad_far}
\ScaDyad_i(\dwi,\dws) & = (\sIdDyad - \dPx_{i}\otimes\dPx_{i}) \cdot\frac{g(-\dws)}{4\pi} \bigcdot{\px'} \dyad{T}_i \\ & \bigcdot{\px''}\Bigg[ \sGreenProp(\dwi) + \sum_{j(\neq i)=1}^\Nnear \left[...\right]_j^{\sGreenProp(\dwi)}\Bigg]. \nonumber
\end{empheq}
}

Finally, since $\dPx_i\approx\dPx_\Cls$ for all particles $i\in\Cls$ we can approximate the far-field scattered field of cluster $\Cls$ as
\begin{align}
\ScaEField_\Cls(\px) & = \sum_{i=1}^{N_\Cls} \ScaEField_i(\px) \nonumber\\
& = \sum_{i=1}^{N_\Cls} \frac{e^{\img k_1 \tPx_i}}{\tPx_i} \Big(\ScaDyad_i(\dw,\dPx_i)\cdot\EField_0 + \sum_{k=1}^\Nfar \ScaDyad_i(\dPx_{\Cls k},\dPx_i)\cdot\ExcEField_{0\Cls k} \Big), \nonumber\\
& \approx \frac{e^{\img k_1 \tPx_\Cls}}{\tPx_\Cls}\Big( \sum_{i=1}^{N_\Cls} \ScaDyad_i(\dw,\dPx_\Cls)\cdot\EField_0 + \sum_{k=1}^\Nfar \sum_{i=1}^{N_\Cls} \ScaDyad_i(\dPx_{\Cls k},\dPx_\Cls)\cdot\ExcEField_{0\Cls k} \Big) \nonumber\\
& = \frac{e^{\img k_1 \tPx_\Cls}}{\tPx_\Cls}\Big( \ScaDyad_\Cls(\dw,\dPx_\Cls)\cdot\EField_0 + \sum_{k=1}^\Nfar \ScaDyad_\Cls(\dPx_{\Cls k},\dPx_\Cls)\cdot\ExcEField_{0\Cls k} \Big),
\label{eq:scatfieldcluster}
\end{align}
%
with $\ScaDyad_\Cls(\dwi,\dws)=\sum_{i=1}^{N_\Cls}\ScaDyad_i(\dwi,\dws)$ the far-field scattering dyad of cluster $\Cls$.


\paragraph{Computing the far-field exciting field}
Let us know compute the far-field exciting field $\ExcEField_{k \Cls}$ from a cluster $\Cls$ to a particle $k$ placed in the far-field region of $\Cls$. By plugging \Eq{eq:foldylaxtwo} into \Eq{eq:excfieldfar}, and under the assumption of far-field incident fields (\Eqs{eq:farincfieldcluster}{eq:farexcfieldcluster}) we get the exciting field from a particle $i\in \Cls$ over particle $j$ as:
%
\begin{align}
\ExcEField_{ki}(\px) & \approx (\sIdDyad - \dPx_{ki}\otimes\dPx_{ki}) \cdot\frac{e^{\img k_1 (\tPx_{ki}+\dPx_{ki}\cdot{\Delta}\px)}}{4\pi\tPx_{ki}}  \sGreenProp(\dPx_{ki}) \bigcdot{\px'} \dyad{T}_j \nonumber \\ &\bigcdot{\px''}\Big[\IncEField + \sum_{k'=1}^\Nfar \ExcEField_{ik'} + \sum_{j(\neq i)=1}^\Nnear \ExcEField_{ij}\Big].
\label{eq:excfield_fartwo}
\end{align}
%
\rev{This equation has the same form as \Eq{eq:scafield_cluster_far3}, and thus we can express it using the far-field scattering dyad defined in \Eq{eq:scatdyad_far} as}
%
\begin{align}
\ExcEField_{ki}(\px) = \frac{e^{\img k_1 (\tPx_{ki}+\dPx_{ki}\cdot{\Delta}\px)}}{\tPx_{ki}} \Big(\ScaDyad_i(\dw,\dPx_i) \cdot\IncEField_0
+ \sum_{k'=1}^\Nfar \ScaDyad_i(\dPx_{\Cls k'},\dPx_i)\cdot \ExcEField_{0\Cls k'}  \Big),
\end{align}
%
which by summing the exciting field of all particles $i\in\Cls$ and following the far-field approximation ($\dPx_{ki}\approx\dPx_{k\Cls}, \forall i\in\Cls$ we get
\begin{equation}
\label{eq:excfield_cluster_far}
\ExcEField_{k\Cls}(\px) \approx \frac{e^{\img k_1 (\tPx_{k\Cls}+\dPx_{k\Cls}\cdot{\Delta}\px)}}{\tPx_{k\Cls}}\Big( \ScaDyad_\Cls(\dw,\dPx_\Cls)\cdot\IncEField_0 + \sum_{k'=1}^\Nfar \ScaDyad_\Cls(\dPx_{\Cls k'},\dPx_\Cls)\cdot \ExcEField_{0\Cls k'} \Big).
\end{equation}
%
Finally, if particle $k$ is itself contained in the near-field of a cluster of particles $\Cls_1$, then it is trivial to compute the exciting field from cluster $\Cls$ to $\Cls_1$ as
\begin{equation}
\ExcEField_{\Cls_1\Cls}(\px) = \sum_{k=1}^{N_{\Cls_1}} \ExcEField_{k\Cls}(\px).
\label{eq:excfield_cluster_far2}
\end{equation}
%

Thus, by grouping the individual particles into $\Ncls$ near-field clusters, and assuming that all clusters and observation point $\px$ lay in their respective far field, we can approximate the Foldy-Lax equation~\eqref{eq:foldylax} as
\begin{equation}
\EField(\px) = \IncEField(\px) + \sum_{\Cls_j=1}^{\Ncls} \ScaEField_{\Cls_j}(\px),
\label{eq:foldylaxcluster}
\end{equation}
%
with $\ScaEField_{\Cls_j}(\px)$ defined by pluging \Eq{eq:excfield_cluster_far2} into \Eq{eq:scatfieldcluster} as
\begin{align}
\ScaEField_{\Cls_j}(\px) = \frac{e^{\img k_1 \tPx_{\Cls_j}}}{\tPx_{\Cls_j}}\Big( &  \ScaDyad_{\Cls_j}(\dw,\dPx_{\Cls_j})\cdot\IncEField_0  \\ & + \sum_{{\Cls_k}(\neq{\Cls_j})=1}^\Ncls \ScaDyad_{\Cls_j}(\dPx_{{\Cls_j}{\Cls_k}},\dPx_{\Cls_j})\cdot\ExcEField_{0{\Cls_j} {\Cls_k}}  \Big),
\label{eq:scatfieldcluster2}
\end{align}
%
with $\ExcEField_{0{\Cls_j} {\Cls_k}}$ the amplitude of the far-field exciting field from cluster $\Cls_k$ to cluster $\Cls_j$. 
%%%%%%%%%%%%%%%%%%%%%%%%%%%%%%%%%%%%%%%%%%%%%%%%%%%%%%%%%%%%%%%%%%%%%%
%%%%%%%%%%%%%%%%%%%%%%%%%%%%%%%%%%%%%%%%%%%%%%%%%%%%%%%%%%%%%%%%%%%%%%
%%%%%%%%%%%%%%%%%%%%%%%%%%%%%%%%%%%%%%%%%%%%%%%%%%%%%%%%%%%%%%%%%%%%%%
\endinput
\section{Statistical average and the coherent field}
%
While the Foldy-Lax equation for clusters in the far-field region~\eqref{eq:foldylaxcluster} can be computed numerically by solving a linear system, this rapidly becomes impractical for a large $\Ncls$. This is actually the case for media in the real world, generally composed by a large number of randomly-distributed particles (and therefore clusters of particles). 
%
Therefore, the electric field $\EField(\px)$ at a point $\px$ is a random function of the time-dependent position $\Px_\Cls$ and properties of clusters $\Pprops_\Cls$ (including e.g. the position of the scatterers within the cluster, their shape, or their permittivity $\sPermittivity$) within the media following a distribution $p(\Px_\Cls,\Pprops_\Cls)$. 

This random field $\EField(\px)$ can be decomposed into the average or coherent field $\CoEField(\px)$ and the fluctuating field $\FlucEField(\px)$ as $\EField(\px)=\CoEField(\px)+\FlucEField(\px)$, where $\CoEField(\px)=\EV{\EField(\px)}$ is the expected value of $\EField(\px)$ and therefore $\EV{\FlucEField(\px)}=0$. 
%
Note that the fluctuating field results into the so-called so-called subjective speckle~\cite{bar2020rendering}; we however focus only on the average coherent field, although our results from Section~\ref{sec:farfield_foldy_clusters} could be used for rendering the fluctuating field too. 

By computing the expected value of the Foldy-Lax equations~\eqref{eq:foldylaxcluster} $\EField(\px)$ can thus express the coherent field as 
\begin{align}
\CoEField(\px) & = \IncEField(\px) & \\ 
& +  \sum_{{\Cls_j}=1}^\Ncls \int_{\Real^3} G(\tPx_{\Cls_j}) \int_\Omega \Big( & \EField_0 \,  \ScaDyad_{\Cls_j}(\dw,\dPx_{\Cls_j}) \\ & & + \sum_{{\Cls_k}(\neq{\Cls_j})=1}^\Ncls \ExcEField_{0{\Cls_j} {\Cls_k}} \, \ScaDyad_{\Cls_j}(\dPx_{{\Cls_j}{\Cls_k}},\dPx_{\Cls_j}) \Big)
\end{align}

$G(\tpx) = \frac{e^{\img k_1 \tpx}}{\tpx}$

%%%%%%%%%%%%%%%%%%%%%%%%%%%%%%%%%%%%%%%%%%%%%%%%%%%%%%%%%%%%%%%%%%%%%%%%%
